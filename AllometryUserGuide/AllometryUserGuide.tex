
\documentclass[11pt,leqno]{article}
\usepackage{amsthm,amsfonts,amsmath,enumerate,amssymb,stmaryrd,newcent} % Typical maths resource packages
\usepackage{graphics}                 % Packages to allow inclusion of graphics
\usepackage{color}                    % For creating coloured text and background
\usepackage{hyperref}                 % For creating hyperlinks in cross references


 \linespread{1.2}
 \setlength{\parskip}{0.2in}
 \setlength{\parindent}{0in}
 \setlength{\oddsidemargin}{0in}
 \setlength{\evensidemargin}{0in}
 \setlength{\textwidth}{6.5in}
 \setlength{\topmargin}{0in}
 \setlength{\textheight}{8in}



 \theoremstyle{plain}
 \theoremstyle{definition}
 \newtheorem{lem}{Lemma}
 \newtheorem{defn}[lem]{Definition}
 \newtheorem{thm}[lem]{Theorem}
 \newtheorem{prop}[lem]{Proposition}
 \newtheorem{cor}[lem]{Corollary}
 \newtheorem{notn}[lem]{Notations}
 \newtheorem{pb}[lem]{Problem}
 \newtheorem{form}[lem]{Formulae}
 \newtheorem*{conj}{Conjecture}
 \newtheorem*{rk}{Remark}
 \newtheorem*{com}{Comment}
 \newtheorem*{ex}{Example}
 \theoremstyle{remark}

 \newcommand{\blem}{\begin{lem} \bf}
 \newcommand{\elem}{\end{lem}}
 \newcommand{\bdefn}{\begin{defn} \bf}
 \newcommand{\edefn}{\end{defn}}
 \newcommand{\bthm}{\begin{thm} }
 \newcommand{\ethm}{\end{thm}}
 \newcommand{\bprop}{\begin{prop} \bf}
 \newcommand{\eprop}{\end{prop}}
 \newcommand{\bcor}{\begin{cor} \bf}
 \newcommand{\ecor}{\end{cor}}
 \newcommand{\bnotn}{\begin{notn}}
 \newcommand{\enotn}{\end{notn}}
 \newcommand{\bpb}{\begin{pb}}
 \newcommand{\epb}{\end{pb}}
 \newcommand{\bform}{\begin{form}}
 \newcommand{\eform}{\end{form}}
 \newcommand{\brk}{\begin{rk}}
 \newcommand{\erk}{\end{rk}}
 \newcommand{\bcom}{\begin{com}}
 \newcommand{\ecom}{\end{com}}
 \newcommand{\bex}{\begin{ex}}
 \newcommand{\eex}{\end{ex}}
 \newcommand{\bpf}{\begin{proof}}
 \newcommand{\epf}{\end{proof}}

 \def \be {\begin{equation}}
 \def \ee {\end{equation}}
 \def \ba {\begin{eqnarray}}
 \def \ea {\end{eqnarray}}
 \def \ba* {\begin{eqnarray*}}
 \def \ea* {\end{eqnarray*}}

 \newcommand{\bmx}{\begin{matrix}}
 \newcommand{\emx}{\end{matrix}}
 \newcommand{\bbmx}{\begin{bmatrix}}
 \newcommand{\ebmx}{\end{bmatrix}}
 \newcommand{\bpmx}{\begin{pmatrix}}
 \newcommand{\epmx}{\end{pmatrix}}
 \newcommand{\bvmx}{\begin{vmatrix}}
 \newcommand{\evmx}{\end{vmatrix}}

 \newcommand{\A}{\mathbb{A}}
 \newcommand{\B}{\mathbb{B}}
 \newcommand{\C}{\mathbb{C}}
 \newcommand{\D}{\mathbb{D}}
 \newcommand{\E}{\mathbb{E}}
 \newcommand{\F}{\mathbb{F}}
 \newcommand{\G}{\mathbb{G}}
 %\newcommand{\H}{\mathbb{H}}
 \newcommand{\I}{\mathbb{I}}
 \newcommand{\J}{\mathbb{J}}
 \newcommand{\K}{\mathbb{K}}
 %\newcommand{\L}{\mathbb{L}}
 \newcommand{\M}{\mathbb{M}}
 \newcommand{\N}{\mathbb{N}}
 %\newcommand{\O}{\mathbb{O}}
 %\newcommand{\P}{\mathbb{P}}
 \newcommand{\Q}{\mathbb{Q}}
 \newcommand{\R}{\mathbb{R}}
 %\newcommand{\S}{\mathbb{S}}
 \newcommand{\T}{\mathbb{T}}
 \newcommand{\U}{\mathbb{U}}
 \newcommand{\V}{\mathbb{V}}
 \newcommand{\W}{\mathbb{W}}
 \newcommand{\X}{\mathbb{X}}
 \newcommand{\Y}{\mathbb{Y}}
 \newcommand{\Z}{\mathbb{Z}}

 %\newcommand{\AA}{\mathcal{A}}
 \newcommand{\BB}{\mathcal{B}}
 \newcommand{\CC}{\mathcal{C}}
 \newcommand{\DD}{\mathcal{D}}
 \newcommand{\EE}{\mathcal{E}}
 \newcommand{\FF}{\mathcal{F}}
 \newcommand{\GG}{\mathcal{G}}
 \newcommand{\HH}{\mathcal{H}}
 \newcommand{\II}{\mathcal{I}}
 \newcommand{\JJ}{\mathcal{J}}
 \newcommand{\KK}{\mathcal{K}}
 \newcommand{\LL}{\mathcal{L}}
 \newcommand{\MM}{\mathcal{M}}
 \newcommand{\NN}{\mathcal{N}}
 \newcommand{\OO}{\mathcal{O}}
 \newcommand{\PP}{\mathcal{P}}
 \newcommand{\QQ}{\mathcal{Q}}
 \newcommand{\RR}{\mathcal{R}}
 \newcommand{\Sp}{\mathcal{S}}
 \newcommand{\TT}{\mathcal{T}}
 \newcommand{\UU}{\mathcal{U}}
 \newcommand{\VV}{\mathcal{V}}
 \newcommand{\WW}{\mathcal{W}}
 \newcommand{\XX}{\mathcal{X}}
 \newcommand{\YY}{\mathcal{Y}}
 \newcommand{\ZZ}{\mathcal{Z}}

 \newcommand{\us}{\underset}
 \newcommand{\os}{\overset}
 \newcommand{\ul}{\underline}
 \newcommand{\ol}{\overline}
 \newcommand{\wh}{\widehat}
 \newcommand{\wt}{\widetilde}

 \newcommand{\la}{\lambda}
 \newcommand{\La}{\Lambda}
 \newcommand{\eps}{\varepsilon}
 \newcommand{\fa}{\forall \,}
 \newcommand{\e}{\exists \,}
 \newcommand{\inc}{\subseteq}
 \newcommand{\empt}{\varnothing}
 \newcommand{\setm}{\setminus}
 \newcommand{\ty}{+\infty}
 \newcommand{\tto}{\longrightarrow}
 \newcommand{\onto}{\twoheadrightarrow}
 \newcommand{\embeds}{\hookrightarrow}
 \newcommand{\imp}{\Longrightarrow}
 \newcommand{\im}{\Rightarrow}
 \newcommand{\f}{\frac}
 \newcommand{\p}{\partial}


 \newcommand{\lbt}{\llbracket}
 \newcommand{\rbt}{\rrbracket}
 \newcommand{\cst}{\mathrm{cst}}
 \newcommand{\Ex}{\mathrm{Ex}}
 \newcommand{\cl}{\mathrm{cl}}
 \newcommand{\interior}{\mathrm{int}}
 \newcommand{\co}{\mathrm{co}}
 \newcommand{\cco}{\overline{\mathrm{co}}}
 \newcommand{\spa}{\mathrm{span}}
 \newcommand{\ran}{\mathrm{ran}}
 \newcommand{\Id}{\mathrm{Id}}
 \newcommand{\id}{\mathrm{id}}
 \newcommand{\sgn}{\mathrm{sgn}}
 \newcommand{\sign}{\backsim}
 \newcommand{\bul}{\mbox{\small $\bullet$}}
 \newcommand{\intg}{\mathrm{int}}
 \newcommand{\ECT}{\mathrm{ECT}}



 \title{Exact Constants in Banach Space Geometry:\\ User's Guide for the Allometry Software}
 \author{Simon Foucart}
 \date{}

 
 \begin{document}

 \maketitle
 
 We describe in this note the MATLAB code that was used to perform the computations of  \cite{paper8}.
We work with a finite-dimensional normed space $\VV$, which is of the following types
 \begin{itemize}
 \item type 0: the $n$-dimensional Euclidean space $\EE_n$,
 \item type 1: the $n$-dimensional sequence space $\ell_p^n$,
 \item type 2: the $(n+1)$-dimensional space $\PP_n$ of algebraic polynomials of degree $\le n$,
 \item type 3: the $(2n+1)$-dimensional space $\TT_n$ of trigonometric polynomials of degree $\le n$,
 \item type 4: the $2$-dimensional space determined by the vertices of its unit ball,
  \item type 5: the $2$-dimensional space determined by the vertices of its dual unit ball. 
 \end{itemize}
 The functions available so far are --- in lexicographic order ---  {\tt \small AbsCdN}, {\tt \small Allo}, {\tt \small Allo}, {\tt \small CanCdN}, {\tt \small CanDual}, {\tt \small CdN}, {\tt \small DualNormAlg}, {\tt \small DualNormPolygon}, {\tt \small DualNormPolygon2}, {\tt \small DualNormTrig}, {\tt \small Mx2Vc}, {\tt \small NormAlg}, {\tt \small NormPolygon}, {\tt \small NormPolygon2}, {\tt \small NormTrig}, {\tt \small Nzed}, {\tt \small Reg2nGone}, {\tt \small Signs}, {\tt \small Signs1}, {\tt \small SymAllo}, {\tt \small Vc2Mx}.
 Note that one can access at any time  a description of a function by typing {\tt help} followed by the name of the function at the MATLAB prompt. For instance, typing\\
{\tt \small >> help CanCdN}\\
returns\\
{\tt \small >>   CANCDN Calculate the canonical condition number\\
  \phantom{>>} CANCDN(V) returns the condition number of the frame formed by the columns of V relatively to its canonical dual frame} 
 
 \section{The three main functions}
 
 The functions {\tt \small CdN}, {\tt \small AbsCdN}, and {\tt \small Allo} are respectively designed to compute 
 \begin{itemize}
 \item the condition number of a system $\ul{v}=(v_1,\ldots,v_N)$ spanning the space $\VV$, relatively to a dual system $\ul{\la}=(\la_1,\ldots,\la_N)$ of linear functional on $\VV$, i.e. the quantity
$$
\kappa_\infty(\ul{v}|\ul{\la}) := 
\sup_{a \in \ell_\infty^N} \f{\big\| \sum\nolimits_{j=1}^N a_j v_j \big\|}{\|a\|_\infty} 
\; \cdot \;
\sup_{v \in \VV} \f{\big\|  (\la(v_j)) \big\|_\infty}{\|v\|},
$$
 \item the absolute condition number of a system $\ul{v}=(v_1,\ldots,v_N)$ spanning the space $\VV$, i.e. the quantity
 $$
 \kappa_\infty(\ul{v}) := \min \big\{
 \kappa_\infty(\ul{v}|\ul{\la}) , \quad
 \mbox{$\ul{\la}$ dual system to $\ul{v}$}
 \big\},
 $$
 \item the $N$-th allometry constant of the space $\VV$, i.e. the quantity
 $$
 \kappa_\infty^N(\VV) := \min \{
 \kappa_\infty(\ul{v}),  \quad
\mbox{$\ul{v}=(v_1,\ldots,v_N)$ spans $\VV$}
 \}.
 $$
 \end{itemize}  
 
 \section{Input arguments}
 
 If a system $\ul{v}=(v_1,\ldots,v_N)$ spanning the $n$-dimensional space $\VV$ needs to be entered, e.g. in {\tt \small AbsCdN}, it is done so 
 via its matrix in a specific basis $\ul{w}=(w_1,\ldots,w_n)$ for $\VV$,
 i.e. via the $n \times N$ matrix
 $$
V= \bbmx
{\rm coef}_{w_1}(v_1) & \ldots & {\rm coef}_{w_1}(v_N)\\
\vdots & \cdots & \vdots\\
{\rm coef}_{w_n}(v_1) & \ldots & {\rm coef}_{w_n}(v_N)
\ebmx.
$$
If in addition a dual system $\ul{\la}=(\la_1,\ldots,\la_N)$  needs to be entered, it is done so via the $n \times N$ matrix
$$
U = \bbmx
\la_1(w_1) & \cdots & \la_N(w_1)\\
\vdots & \cdots & \vdots\\
\la_1(w_n) & \cdots & \la_N(w_n)
\ebmx .
$$
Note that the number of elements in the system $\ul{v}$ and the dimension of the space $\VV$ are implicitly  entered.
If we want to calculate the $N$-th allometry constant of a $n$ -dimensional space $\VV$, however, we need to input the integers $N$ and $n$.
For example, we would type\\
{\tt \small >> Allo(3,4,0)}\\
{\tt \small >> Allo(3,4,1,1)}\\
to compute the $4$-th allometry constant of the $3$-dimensional space $\EE_3$ --- space of type 0 --- and then the $4$-th allometry constant of the $3$-dimensional space $\ell_1^3$ --- space of type 1, specifying $p=1$ as the fourth argument.
When working with polygonal spaces --- spaces of type 4 and type 5 --- we provide the vertices of the (dual) unit ball as fourth argument. For instance,\\
{\tt \small >>Hex=[1 1/2 -1/2 -1 -1/2 1/2; 0 sqrt(3)/2 sqrt(3)/2 0 -sqrt(3)/2 -sqrt(3)/2];}\\
{\tt \small >>[Allo(2,2,4,Hex),Allo(2,3,4,Hex)]}\\
returns {\tt \small 1.5000} and {\tt \small 1.3333}, which are the $2$-cd and $3$-rd allometry constants of the `hexagonal' space and
believed to be maximal among all $2$-dimensional spaces. 

 
 \section{Output arguments}
 
 If only one output argument is requested, then the result is the numerical value of the quantity invoked by the function.
 If more than one output argument is requested, then optimal systems are also returned.
 For example, typing \\
 {\tt \small >> [k,OS,OD]=Allo(2,3,0)}\\
 returns the value of the $3$-rd allometry constant of the Euclidean plane $\EE_2$,
 together with a best conditioned $3$-frame {\tt \small OS} and an optimal dual frame {\tt \small OD}.
 

%To avoid confusion, we will run the indices $h$ in $1:2^{N-1}$, $i,k$ in $1:n$,  and $j,l$ in $1:N$.  

  
 
 \begin{thebibliography}{99}
 

  \bibitem{paper8} Foucart, S.,
  Allometry constants of finite-dimensional spaces: 
theory and computations,
 Preprint. 

  \end{thebibliography}   

 
 \end{document}
